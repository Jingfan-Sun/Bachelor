\chapter{外文文献翻译}
%==============================================================
%英文文献
  \section{外文文献原稿}

  \includepdf[pages={1-17}]{data/Paper.pdf}

%==============================================================
  \section{中文翻译译稿}
  
  模块化多电平换流器的运行、控制和应用
  
  \textbf{摘要:}\quad  模块化多电平换流器(MMC)对于中高压电力换流系统来说正在变成一个越来越重要的课题。在过去的几年里,重要的研究指出了伴随着MMC运行和控制所产生的技术挑战。在这篇文章里,我们综述了MMC运行基础并讨论了控制所面临的挑战和先进的控制策略和趋势。最后,强调了MMC的应用和挑战。
  
  \textbf{关键词:}\quad 电容电压平衡,环流控制,高压直流{HVDC}输电,模块化多电平换流器{MMC},模块化技术,冗余,变速传动系统
  
  \subsection{导言}
  
  模块化多电平换流器(MMC)正成为中高级电压应用中最有吸引力的多电平换流器拓扑,特别是对电压源换流器高压直流输电系统。和其他多电平换流器拓扑相比,MMC的显著特点包括:1)、模块化和可测量性满足任何电压等级需要;2)、高效,对高压应用有特别重要性;3)、优秀的谐波表现,特别是在高压应用当中,在这种应用中通过组合大量相同的低压子模块,减小了被动滤波器的大小;4)、不需要直流连接电容。
  
  在过去几年中,有大量的研究致力于找到MMC控制和运行所带来的技术挑战并扩展其应用。这篇综述的主要目的是为MMC提供一个更好的理解和针对各种应用的技术命题。这篇文章全面总结了关于MMC运行、建模、控制和模块化技术最近的成果,同时强调了MMC的广泛应用和伴随的挑战。
  
  文章的其余部分按如下方式展开。第二部分介绍MMC电路拓扑和能应用于换流器设计的各种子模块。接下来是关于MMC模块化技术、设计局限和各种运行问题最新成果的综述,包括电容器电压平衡和环路电流控制都在第三章中呈现。第四章包括MMC在特殊环境下运行和控制的最新发展,例如不平衡电网条件下的运行和容错运行。第五章介绍MMC的广泛应用和随之而来的技术挑战。第六章总结了全文。
  
  \subsection{MMC拓扑}
  
\begin{figure}[h]
\begin{minipage}[t]{0.5\linewidth}
\centering
\includegraphics[width=3.2in]{images/Paper_Fig_1.png}
\setcaptionwidth{2.5in}
\caption{MMC示意图}
\end{minipage}%
\begin{minipage}[t]{0.5\linewidth}
\centering
\includegraphics[width=3.2in]{images/Paper_Fig_2.png}
\setcaptionwidth{2.5in}
\caption{ 不同SM拓扑:\quad(a)半桥\quad(b)全桥\quad(c)双钳制\quad(d)三级FC\quad(e)三级NPC和(f)五级跨接SM}
\end{minipage}
\end{figure}
  
  图3.1所示是三相MMC的示意图,这个MMC由每相两臂组成,每臂由$N$个序列连接、名义上独立的SM和一个序列电感$L_0$组成。尽管控制每一臂中的SM长生需要的交流相电压,臂电感器抑制了臂电流中的高频分量。三相臂中的上(下)臂表示成$"p"("n")$。
  
  图3.1中MMC的SM可以从如下回路来理解:
  \begin{enumerate}[1)]
  \item 半桥回路或斩波单元(chopper-cell):如图3.2(a)所示,半桥SM的输出电压或等于其电容电压$v_c$(开/接入状态),或等于$0$(关/旁路状态),取决于开关对的开关状态,例如$S1$和$S2$。
  \item 全桥回路或桥臂单元(bridge-cell):如图3.2(b)所示,全桥SM的输出电压或等于其电容电压$v_c$(开/接入状态),或等于$0$(关/旁路状态),取决于四个开关$S1$到$S4$的状态。因为全桥SM的半导体数是半桥SM半导体数的两倍,基于全桥SM的MMC的能耗和费用都高于基于半桥SM的MMC。
  \item 双钳制回路(clamp-double):如图3.2(c)所示,双钳制回路SM由两个半桥SM、两个额外的二极管和一个与有反向二极管并联的绝缘栅双极型晶体管(IGBT)组成,正常操作中,开关$S5$一直开启,双钳制回路SM等价于两个串联的半桥SM,与同电压等级数量的半桥和全桥MMC相比,双钳制回路MMC的半导体能耗比半桥MMC高但比全桥MMC少。
  \item 三级换流器回路:如图3.2(d)和(e)所示,三级SM由三级中性点钳制(neutral-point-clamped: NPC)或三级飞跨电容(flying capacitor: FC)换流器。三级FC MMC和半桥MMC半导体能耗相同。相比之下,三级NPC MMC的半导体能耗比半桥MMC高但比全桥MMC低。从制造和控制的角度,这种SM回路吸引力不大。
  \item 五级跨接回路(cross-connected):如图3.2(f)所示,一个五级跨接回路由两个半桥SM背靠背连接两个有反向并联二极管的IGBT组成。它的半导体能耗与双钳制回路相同。
  \end{enumerate}
  
  表3.1展示了有关不同SM回路电压等级、直流侧短路容错能力和电能损耗的比较。直流侧短路问题是MMC-HVDC系统的主要挑战之一,这部分会在第五章A中讨论。在这所有SM回路中,半桥SM是适用于MMC最受欢迎的SM。因为其中只有两个开关器件,所以减少了元件数并且提高了效率。在此之后,基于半桥SM的MMC开始被人们关注。需要注意的是,只有很少的换流器结构源自MMC拓扑。本文着重讨论图3.1所示的称为双星(double-star)的MMC结构。
  
\begin{table}[h]
\centering
\begin{tabular}{cccc}
\toprule
SM回路 & 电压等级 & 直流容错 & 能耗\\
\midrule
半桥 & $0, v_c$ & 否 & 低 \\ 
全桥 & $0, +v_c$ & 是 & 高 \\ 
双钳制 & $0, v_{C1}, v_{C2}, (v_{C1} + v_{C2})$ & 是 & 中 \\ 
三级FC & $0, v_{C1}, v_{C2}, (v_{C1} - v_{C2})$ & 否 & 低 \\ 
三级NPC & $0, v_{C2}, (v_{C1} + v_{C2})$ & 否 & 中 \\ 
五级跨接 & $0, v_{C1}, v_{C2}, +(v_{C1} + v_{C2})$ & 是 & 中 \\ 
\bottomrule
\end{tabular}
\caption{不同SM回路比较}
\end{table}
  
  \subsection{MMC的模块化、设计、控制和建模}
  
  \subsubsection{模块化技术}
  
  针对MMC开发/提出的基于单个参考波形的不同的脉宽调制(PWM)技术包括:
  %%%%%%%%%%%%%%%%%%%%%%%%%%%%%%%%%%
  \begin{enumerate}[1)]
  \item 载波层叠PWM技术(CD-PWM):这种技术需要$N$个独立三角载波波形关于$0$坐标轴对称出现。相电压参考波形和载波的比较产生了所需的开关输出相电压等级。对应于三角形载波的电压变换由一个特定SM的开启/旁路状态决定。基于载波波形的相变换,这一技术被进一步划分为:a)同相层叠(PD);b)正负反向层叠(POD);c)交替反向层叠(APOD),分别如图3.3(a)-(c)所示。使用这些技术的不足之处包括SM电容上电压波动的不等分布和大量环流。为了提高交流侧电压的谐波失真,使用简单载波旋转技术、修正载波旋转技术或信号旋转技术来所有SM电容器的电压平衡。尽管采用了SM电容器电压平衡技术,输出电压有一个相对较高的总谐波失真(THD)。为了提高这些技术的表现,提出了一种有关SM电容电压平衡技术的PD PWM技术。在这种基于PD载波波形的技术中,有关三角形载波的电压变换不再针对某个特定的SM。在这种技术中,参考波形和载波波形的比较产生出$(N+1)-level$波形,进而分别决定了插入上下桥臂中SM的数量。取决于桥臂中的电流方向河SM电容器的电压状态,在上(下)桥臂中插入除$N$个SM之外特定数量的SM来最小化SM电容器电压间的差异。Mei等人提出了一种带选择性回路偏置映射法的PD PWM技术来平衡SMSM电容器。这种方法实现了使用下列反馈的载波旋转:a)最大/最小SM电容电压和b)桥臂电流方向。这种技术的优点包括:a)不需要额外参考信号来控制SM电容器和b)即便在大量SM情况下易于用简单的现场可编程门阵列(FPGA)实现。
\begin{figure}[H]
\begin{minipage}[t]{0.5\linewidth}
\centering
\includegraphics[width=3.2in]{images/Paper_Fig_3.png}
\end{minipage}
\begin{minipage}[t]{0.5\linewidth}
\centering
\includegraphics[width=3.2in]{images/Paper_Fig_4.png}
\end{minipage}
\setcaptionwidth{\linewidth}
\caption{多电平载波:\quad (a) PD, \quad (b) POD, \quad (c) APOD, \quad(d) saw-tooth, and (e) 相变换载波}
\end{figure}
  %%%%%%%%%%%%%%%%%%%%%
  \item 次谐波技术:在这种技术中,每相有$2N$个独立的载波,锯齿波或三角波,之间有$\theta=360^{\circ}/2N$相位差,如图3.3(d)和(e)所示。假设对PD PWM和次谐波技术有相同数量的开关变换,PD PWM技术产生更好地线到线电压THD。
  \end{enumerate}
  
  另外,基于使用多参考波形有几种模块化技术。这些模块化技术包括:
  \begin{enumerate}[1)]
  \item 直接模块化:在这种模块化技术中,$j$相的上下桥臂电压由两个互补的正弦参考曲线控制,如下所示:
\begin{align*}
n_{p,j,ref} &= N\frac{\frac{V_{dc}}{2}-v_{j,ref}}{V_{dc}}\tag{1a}\\
n_{p,j,ref} &= N\frac{\frac{V_{dc}}{2}+v_{j,ref}}{V_{dc}}\tag{1b}
\end{align*}
  其中$v_{j,ref}$代表参考输出电压,$n_{p,j,ref}$和$n_{n,j,ref}$是上下桥臂中开启SM数量的参考波形。(1)中的参考波形和PD载波波形比较,在$0$到$N$中波动,来决定上下桥臂中需要开启的SM数量。直接模块化技术的主要缺点在于存在环路电流,增加了换流器能耗和组件的额定值。
  \item 间接模块化:在这种技术中,$j$相上下桥臂的参考波形如下给出:
\begin{align*}
n_{p,j,ref} &= N\frac{\frac{V_{dc}}{2}-v_{j,ref}-v_{reg,j}^\Sigma-v_{reg,j}^{circ}}{\sum^N_{i=0}v_{cp,i,j}}\tag{2a}\\
n_{p,j,ref} &= N\frac{\frac{V_{dc}}{2}+v_{j,ref}-v_{reg,j}^\Sigma-v_{reg,j}^{circ}}{\sum^N_{i=0}v_{cn,i,j}}\tag{2b}
\end{align*}
  其中$v_{cx,i,j}$表示$j$相桥臂$x$中$SM-i$的电容电压,$v_{reg,j}^\Sigma$和$v_{reg,j}^{circ}$用来控制$j$相的总能量并分别平衡桥臂之间的能量。类似于直接模块化技术,参考波形和PD载波波形比较,在$0$到$N$中波动,来决定上下桥臂中需要开启的SM数量。这种技术可以进一步细分成:
  \begin{enumerate}[a)]
  \item 闭环控制:闭环控制中,(2)中的$\sum^N_{i=0}v_{cp,i,j}$项基于实际测量电容电压计算,进一步,$v_{reg,j}^\Sigma$和$v_{reg,j}^{circ}$从$j$相桥臂电容储存的闭环控制总能量和桥臂间能量平衡分别获得。每个桥臂间储存能量的平衡暂时取决于基频正弦环路电流。这种技术的优点在于i)平均SM电容电压的控制,使得能在高电压等级和低输出电压下运行,和ii)上下桥臂间能量平衡的控制。
  \item 开环控制:开环控制中,(2)中的$\sum^N_{i=0}v_{cp,i,j}$项基于估计测量电容电压计算。另外,$v_{reg,j}^\Sigma=0$并且估计$v_{reg,j}^{circ}$来降低环路电流的谐波并保证换流器的稳定控制。这种估计由求解描述换流器动态的方程获得,使用测量得到的输出电流和直流连接电压。这种技术的优点在于不需要电压传感器,控制简单迅速。尽管如此,主要缺点在于难以准确估计描述系统动态所需要的真实参数。
  \end{enumerate}
  %%%%%%%%%%%%%%%%%%%%
  \item 相变换载波PWM技术(PSC PWM):在这种技术中,MMC的每个SM独立控制,Sm的电压平衡任务分为平均控制和平衡控制。每一个SM上下桥臂中的的参考波形如下给出:
\begin{align*}
m_{p,i,j} &= N\frac{\frac{V_{dc}}{2}-\frac{v_{j,ref}}{N}+v_{a,j}+v_{b,i,j}}{v_{cp,i,j}}\tag{3a}\\
m_{p,i,j} &= N\frac{\frac{V_{dc}}{2}+\frac{v_{j,ref}}{N}+v_{a,j}+v_{b,i,j}}{v_{cn,i,j}}\tag{3b}
\end{align*}
  其中$v_{a,j}$和$v_{b,i,j}$分别是平均和平衡控制器输出。平均和平衡技术分别控制每一相桥的平均SM电容器电压。每一个SM电压参考波形和三角载波的比较产生相应SM的开关信号。每一个桥臂的三角形载波波形基于次谐波技术实现。这项技术的主要缺点在于随着SM数量的增加实现难度加大并且在特定运行情况下存在不稳定性。通过基于上下桥臂中电容电压的不同在参考电压中引入另一个概念,桥臂平衡控制,可以改善后一种缺点。
  \end{enumerate}
  %%%%%%%%%%%%%%%%%%%%%%%%%%%%%%%%%%
\begin{table}[h]
\begin{center}
\small
\begin{tabular}[c]{ccccc}
\toprule
性质 & 改进PD-PWM & 直接模块化 & 间接模块化 & PSC-PWM\\
\midrule
参考波形数 & $1$ & $2$ & $2$ & $2N$\\ 
稳定所需额外控制器数 & 否 & 否 & 是 & 是\\ 
是否存在环路电流 & 是 & 是 & 否 & 否 \\ 
实现难度 & 中 & 中 & 低(开环)-中(闭环) & 高(随着$N\uparrow$) \\ 
\bottomrule
\end{tabular}
\caption{模块化策略比较}
\end{center}
\end{table}
  
  表3.2给出了前述模块化策略的简单比较,MMC控制策略在图3.4的表格中总结。除了前述的PWM技术,还提出了一种SHE-PWM技术,其中限定开关模式来减少输出电压波形的低次谐波。开关模式被计算并储存在各种模块化索引和输出电压相角的查询表中。
  
  基于基频开关的模块化技术被提出和研究。提出了一种最近层控制(NLC)模块化技术,其中选择离所需电压波形最近的电压等级。相比于SHE-PWM,NLC技术易于实现,所需计算能力更少,相比于PWM技术使用的开关频率更低。文献[35]中提到的技术基于给定SM的固定脉冲形式来保持每一个SM中存储能量的稳定性,而不用测量电容器电压或任何反馈控制,去除任何强制模块化索引和输出电压相角的特定的输出电压谐波。[36]提出的技术最优化了脉冲形式来减小输出电压的谐波失真。基频开关技术的主要优势在于减小开关频率并减小输出电压的THD的同时对输出电压频率没有任何限制。这点和PWM技术不同,其载波频率限制了输出电压的频率。
\begin{figure}[H]
\centering
\includegraphics[width=1.05\textwidth]{images/Paper_Fig_5.png}
\setcaptionwidth{\linewidth}
\caption{MMC使用的不同PWM技术概览}
\end{figure}
  
  \subsubsection{SM电容电压平衡}
  
  类似于其他多级换流器技术,MMC需要有功电压平衡策略来平衡和保持SM电容器电压在$V_{dc}/N$。Deng和Zhen在[37]中提出一种电压平衡策略,使用相变换载波PWM(PSC-PWM)来控制MMC桥臂电流中的高频分量。给SM每个桥臂加载合适的PWM脉冲来平衡电容器电压。这增加了控制的简单性并减少了传感器数量。Hagiwara和Akagi在[30]中提出一种电压平衡策略,针对每个SM的闭环控制。在[11]中,开发了一种MMC控制的预测性策略,其中SM电容器电压基于预先定义好的损耗方程平衡。最广泛接受的电压平衡策略是基于排序法(sorting method)[8], [38]-[40]。为了实现基于排序法的电容电压平衡,需要测量和排序SM每一个桥臂的电容器电压。如果$N$个SM对应的桥臂$n_{p,j}(n_n,j)$中上(下)桥臂电压为正,有最低电压的SM被选出开启。结果,对应开启的SM电容器充电,电压升高。如果$N$个SM对应的桥臂$n_{p,j}(n_n,j)$中上(下)桥臂电压为负,有最高电压的SM被选出开启。结果,对应开启的SM电容器放电,电压下降。不考虑上(下)桥臂的电流方向,如果一个桥臂中的Sm被旁路,对应的电容器电压保持不变。尽管排序法保证电容电压平衡在MMC运行要求之下,却在SM中带来了不必要的开关变换。尽管在两个连续控制期中所需运行的SM数保持不变,可能会发生SM的开启或旁路。这个结果不是很理想,因为增加了开关频率相应的增加了能耗,特别是对于高压系统来说。提出/研究的用来减小MMC开关频率的方法主要基于:
  \begin{enumerate}[1)]
  \item 结合相变换载波PWM策略的闭环修正排序法,SM的开启/旁路状态基于电容电压测量[39], [41]。在这种方法中,每个控制环中只有有限数量的SM参加排序,即在每个控制环并基于需要的电压等级,如果需要在每个桥臂中开启(关断)额外的SM,只有关状态(开状态)的Sm需要被考虑排序和开关。
  \item 结合选择性谐波消除PWM技术的开环控制[35]。
  \item 混合平衡策略,结合预测误差排序法和传统的电压排序算法[42]。这种策略基于排序向前一步预测电容电压和其有名值之间的绝对误差来排序。每一个控制期内,有最小预测电压误差的SM被选择开启。
  \item 基频平衡策略,在预先制定的相角下基于传统方法排序。
  \item 一种最优化电容电压平衡策略,排序之前先调整测量之后的SM电容电压。这种策略关注电容电压超过特定电压限制的SM,其他SM的开关状态保持相同。引入保持因子并和关闭状态的电容器电压相乘,这些电容器电压超过电压限增加了在下一个控制环内被开通/旁路的概率。
  \item 一种预测算法来计算和分配储存在SM电容器中的电量[44]。
  \end{enumerate}
  
  \subsubsection{数学模型}
  
\begin{figure}[H]
\centering
\includegraphics[width=1.05\textwidth]{images/Paper_Fig_6.png}
\setcaptionwidth{\linewidth}
\caption{MMC电路图}
\end{figure}
  图3.5所示为一个基于半桥SM的三相MMC电路图。与图3.1相比,有一个额外的桥臂电阻$R_0$,用来模拟MMC每个桥臂的能耗。
  
  本章建立的MMC数学模型基于文献[8], [11], [45-47]中使用最广泛的模型。
  
  在图3.5中的MMC中,$j$相上下桥臂电流,$j=a, b, c, i.e, i_{p, j}$and$i_{n, j}$表示为:
\begin{align*}
i_{p, j} = \frac{i_{dc}}{3}+i_{circ, j}+\frac{i_j}{2}\tag{4a}\\
i_{n, j} = \frac{i_{dc}}{3}+i_{circ, j}-\frac{i_j}{2}\tag{4b}
\end{align*}

  其中$i_{circ, j}$表示$j$相环路电流,$i_j$是交流侧$j$相电流,$i_{dc}$是直流侧电流。基于(4),环路电流为:
\begin{align*}
i_{circ, j} = \frac{i_{p, j}+i_{n, j}}{2}-\frac{i_{dc}}{3}\tag{5}
\end{align*}

  由$j$相MMC动态行为给出的数学公式为:
\begin{align*}
V_{dc}-v_{p, j} &= L_0\frac{di_{p, j}}{dt}+R_0i_{p, j}+v_j+v_{cm}\tag{6a}\\
V_{dc}-v_{n, j} &= L_0\frac{di_{n, j}}{dt}+R_0i_{n, j}-v_j-v_{cm}\tag{6b}
\end{align*}

  其中$v_{p, j}$和$v_{n, j}$表示MMC$j$相上下桥臂电压,$v_j$和$v_{cm}$分别表示基本和共模电压分量。从(6a)中减去(6b)并从(4)中替换出$i_{p, j}$和$i_{n, j}$,MMC相电压表示为:
\begin{align*}
v_{j}+v_{cm} = \frac{v_{n, j}-v_{p, j}}{2}-\frac{R_0}{2}i_j-\frac{L_0}{2}\frac{di_j}{dt}\tag{7}
\end{align*}

  进一步,将(6a)和(6b)相加并从(5)中减去$i_{circ, j}$,MMC环流内部动态表达为:
\begin{align*}
L_0\frac{di_{circ, j}}{dt}+R_0i_{circ, j} = \frac{V_{dc}}{2}-\frac{v_{n, j}+v_{p, j}}{2}-R_0\frac{i_{dc}}{3}\tag{8}
\end{align*}

  MMC的$j$相上下桥臂电压也可以描述为:
\begin{align*}
v_{p, j} &= n_{p, j}v_{cp, j}\tag{9a}\\
v_{n, j} &= n_{n, j}v_{cn, j}\tag{9b}
\end{align*}
  
  其中$v_{cp, j}$和$v_{cn, j}$分别是上下桥臂独立SM电容器电压。方程(9)基于有功电容器电压策略平衡的假设并保持MMC每个桥臂中所有SM的电容器电压相等,例如$v_{cxi, j, k} = v_{cx, j, k}$对$\forall i\in{1, 2, ..., N}$。将(9)中的$v_{p, j}$和$v_{n, j}$带入(7)和(8),得到下列表达式:
\begin{align*}
v_{j} + v_{cm} &= \frac{n_{n, j}v_{cn, j}-n_{p, j}v_{cp, j}}{2}-\frac{R_0}{2}i_j-\frac{L_0}{2}\frac{di_j}{dt}\tag{10a}\\
L_o\frac{di_{circ, j}}{dt} + R_0i_{circ, j} &= \frac{V_{dc}}{2} - \frac{n_{n, j}v_{cn, j}-n_{p, j}v_{cp, j}}{2} - R_0\frac{i_{dc}}{3}\tag{10b}
\end{align*}

  另外,
\begin{align*}
P_{dc} = P_{ac} + P_{loss} \quad \Rightarrow \quad V_{dc}i_{dc} = \sum_{j = a, b, c}v_ji_j + P_{loss}\tag{11}
\end{align*}
其中$P_{loss}$表示换流器的能耗。

  MMC每一个SM电容电压由每臂的功率建模。每臂的功率由下式给出:
\begin{align*}
p_{p, j} &= v_{p, j}i_{p, j} = n_{p, j}v_{cp, j}i_{p, j}\tag{12a}\\
p_{n, j} &= v_{n, j}i_{n, j} = n_{n, j}v_{cn, j}i_{n, j}\tag{12b}
\end{align*}
  
  MMC每相桥臂上的功率也可以表示为:
\begin{align*}
p_{p, j} = \frac{dW_{p, j}}{dt} &= \frac{d(\frac{N}{2}C_{SM}v^2_{cp, j})}{dt}\\
&= v_{cp, j}NC_{SM}\frac{dv_{cp, j}}{dt}\tag{13a}\\
p_{n, j} = \frac{dW_{n, j}}{dt} &= \frac{d(\frac{N}{2}C_{SM}v^2_{cn, j})}{dt}\\
&= v_{cn, j}NC_{SM}\frac{dv_{cn, j}}{dt}\tag{13b}\\
\end{align*}
其中$C_{SM}$表示SM电容量。基于(12)和(13),每个SM电容器电压波动的动态可以表示为:
\begin{align*}
\frac{dv_{cp, j}}{dt} &= \frac{i_{p, j}}{NC_{SM}}n_{p, j}\\
&= \frac{1}{NC_{SM}}(\frac{i_{dc}}{3} + i_{circ, j} + \frac{i_j}{2})n_{p, j}\tag{14a}\\
\frac{dv_{cn, j}}{dt} &= \frac{i_{n, j}}{NC_{SM}}n_{n, j}\\
&= \frac{1}{NC_{SM}}(\frac{i_{dc}}{3} + i_{circ, j} + \frac{i_j}{2})n_{n, j}\tag{14b}\\
\end{align*}  
  
  (4), (5), (11)和(14)式提供了一个MMC一般化的动态模型,可以用于控制目的。类似于MMC的这种一般化动态模型,参考文献[27]和[28]将每个桥臂中电容电压的累加组成的MMC动态模型看成是一个静态变量,而不是独立的SM电容电压。

  \subsubsection{MMC设计限制}
  MMC的设计包括选择电容、电感和SM的数量,基于一些性能指标包括桥臂电流波动、短路电流、电容电压波动、可靠性和能耗。
  \begin{enumerate}[1)]
  \item \emph{电容电感选型:}桥臂电感$L_0$作为滤波器来减小桥臂电流中的高频谐波同时也限制了直流侧短路电流。因此,桥臂电感的大小取决于滤波选择和短路电流限制[5], [51], [52]。
  
  SM电容的选型基于对大小/能耗和电压震荡的权衡。在很大范围的运行环境中,SM电容电压震荡在[55]-[59]中得到分析。对于SM电容电压$\delta v_{c, pp}$考虑一个许可的峰峰谐波等级,基于[55]的元电容取决于:
\begin{align*}
C_{SM} = \frac{P}{3NmV_c\delta c_{c, pp}\omega cos\theta}(1-(\frac{mcos\theta}{2})^2)^{\frac{3}{2}}\tag{15}
\end{align*}
其中$C_{SM}$是SM电容,$P$是有功功率,$V_c$是SM电容器的标么电压,$m$是调节指数,而$cos\theta$是功率因数。

  基于存储能量和功率流动的关系,[61]提出一个SM电容器选型方法。
  \item \emph{功率损耗计算:}相比于二级VSC系统,半导体损耗大于1\%[13],MMC的半导体损耗可以潜在地降低到低于1\%。
  
  主要有两种方法来评估MMC的半导体损耗。第一种取决于仿真模型和实时仿真数据[40], [63]-[67]。尽管这种方法的计算压力很大,却可以提供精确的结果而不用考虑模块化/控制策略,电压等级和SM电路拓扑。[62]和[63]介绍的另一种方法,基于分析性模型可以潜在地降低计算时间。可是,保证精度方面是一个挑战因为这种方法基于理想假设和特殊的模块化/控制策略。
  \item \emph{可靠性:}MMC因为其模块化的结构,可以提升冗余SM结构中的容错限[12], [68],进而提升其可靠性。可是,在特定的容错限之下,控制硬件限制了换流器的稳定性。
  \end{enumerate}
  
  \subsubsection{环路电流控制}
  
  流经MMC三相桥臂的环路电流源自三相桥臂中的电压差异[53], [70]并且包含频率是基频两倍的负序分量[70]。环路电流对于交流侧电压电流没有任何影响。可是,如果不正确地控制,他们会增加相电流的峰值结果将增加换流器的能耗和SM电容电压的震荡大小。环路电流在[70]-[72]中被建模和分析。[71提出一种环路电流模型基于控制环路电流为电流控制电压源。这可以接着被用于控制电压来降低环路电流分量的影响。
  
  为了控制环路电流,文献[11], [28]-[30], [41], [45], [46], [57], [73], [74]中提出了各种各样的技术。[28]-[30]中的间接建模技术在前述章节中已经介绍。Harmefors等人使用有功电阻(合适的控制器)来控制环路电流,包括对桥臂电阻的估计。基于双线频$acb-dp$变压器,Tu等人[41]通过控制他们一对PI控制器的的$dp$部分来减小环路电流。Debnath和Saeedifard[45]等人致力于降低通过比例共鸣(PR)控制器的环路电流的交流分量。Yang[57]等人提出一种改进的开关函数来消除用于STATic和COMpensator(STATCOM)中的MMC环路电流。[11]提出一种基于模型的预测控制(MPC)来降低环路电流分量。主要缺点在于计算量大,尤其对于有大量SM的MMC来说。
  
  \subsubsection{SM电容电压震荡消除技术}
  
  [55]深入研究了SM电容电压震荡, [57]-[59]主要研究基频和二次谐波分量。基于桥臂功率,[76]提出使用合适的二次谐波分量来降低SM电容电压的震荡大小。使用二次和四次谐波分量[56]中的方法最优化了每个桥臂中的能量变化。Debnath和Saeedifard[45]提出一种闭环控制策略来减小电容电压波动的二次谐波分量并且数学地证明了所提出的方法简介见底了电容电压的震荡大小。
  
  Engel和Doncker[56]提出最优化方法通过保持桥臂电流的rms值来降低电容电压波动,进而限制了换流器的功率损失。
  
  除了SM电容电压波动的减小策略,文献[49]和[79]变形电容电压波动来最大化运行区域(有功率极限定义)。
  
  \subsubsection{SM电容预充电和启动过程}
  
  MMC启动时,在正常工作之前需要提前充电到相等的特定电压等级。为了减小冲击电流和启动时间,最好是一种快速平滑启动。[55], [80]-[86]从去能量化的角度研究了MMC中SM电容器的预充电和启动过程。启动过程经两步进行:1)分别充电直流侧电容器和经过二极管桥整流器的SM电容器至$V_{dc}$和${\frac{V_{dc}}{2N}}$,假设所有SM电容器插入直流侧并且涌入电流被交流侧的串联电阻限制,2)旁路交流测电阻接着从$2N$到$N$逐渐减少开通SM的数量,进而让电容从$\frac{V_{dc}}{2N}$到$\frac{V_{dc}}{N}$。
  
  \subsection{特殊情况下MMC的运行}
  
  \subsubsection{非平衡电网条件}
  
  建模和仿真MMC的主要技术文献假设系统处于平衡状态[1]-[7], [9], [10], [12]。非平衡条件下MMC的控制在文献[8], [9], [87]-[89]中有讨论。在非平衡电网条件下,主要的控制目标是:i)通过限制负序分量来平衡交流侧电压,ii)调节电网直流总线电压,iii)控制环路电流和SM电容器电压。
  
  Saeedifard和Iravani[8]提出了一种普遍的PWM策略来在非平衡电网条件下控制MMC,这种情况下,MMC动态可以被视为两个解耦的子系统,正序和负序子系统;每个子系统可以被分别控制。
  
  Tu等人[87]提出一种直流电压波动限制的控制器来去除非平衡电网条件下直流侧零序分量并保持电网直流总线电压恒定。可是,在非平衡电网电压条件下,在实际功率器件中有双线频率波动。Guan和Xu在[9]中提出零序交流电压控制器,和正负序交流电压控制器,来在非平衡电网条件下运行MMC-HVDC系统通过/不通过接口变压器。
  
  \subsubsection{容错运行}
  
  如前面所提到的,MMC的模块化设计提高了其冗余和容错。在任何原件/SM失效的前提下,失效的SM需要被检测出来并旁路。当一个开环错误出现时,MMC的输出电压和电流紊乱。进一步,失效的SM电容器电压升高,导致更大的破坏。
  
  \subsection{应用}
  
\begin{figure}[H]
\begin{minipage}[t]{0.5\linewidth}
\centering
\includegraphics[width=3.2in]{images/Paper_Fig_7.png}
\setcaptionwidth{2.5in}
\caption{直流侧短路错误电流路径:\quad (a) \quad 半桥;\quad (b) \quad 全桥;\quad (c) \quad 双钳制;\quad (d) \quad 五级跨接SM}
\end{minipage}%
\begin{minipage}[t]{0.5\linewidth}
\centering
\includegraphics[width=3.2in]{images/Paper_Fig_8.png}
\setcaptionwidth{2.5in}
\caption{直流短路时MMC系统的等效电路}
\end{minipage}
\end{figure}  
  
  \subsubsection{HVDC系统}
  
  MMC最初是为了HVDC系统而提出,变成了HVDC系统最可靠的VSC形式。采用传统半桥SM的MMC-HVDC系统的一个主要挑战之一是缺少直流侧电流错误处理能力。这个问题非常严重,尤其是对于没有过载能力的HVDC系统来说。现有的干预和处理MMC-HVDC系统直流侧短路故障的方法总结如下:
  \begin{enumerate}[1)]
  \item 打开交流侧CB。这种方案并不是很快速因为它需要几个循环,比如两到三个循环。结果是,组成非控制整流器的MMC惯性二极管需要忍受几个循环高故障电流。
  \item 采用直流侧CB。尽管用于HVDC的固态直流CB怎年来得到开发,但这项技术并不是很成熟费用也比较高。
  \item 在HVDC换流器结构中嵌入直流故障处理能力。
  \end{enumerate}
  
  \subsubsection{变速伺服}
  
  MMC在中压变速伺服中的应用比其他多级换流器例如NPC和串联H桥换流器有更大的优势。可是,这种应用有其自身独有的控制挑战。主要的挑战在于在低频情况下SM电容器电压有很大的波动量。
  
  SM电容电压的峰峰波动量由下式给出[83]:
\begin{align*}
\delta v_{c, pp} = \frac{I_0}{2C_{SM}\omega}(1-(\frac{mcos\theta}{2})^2)^{\frac{3}{2}}\tag{16}
\end{align*}
其中$i_{circ, j}\approx 0$,$I_0$是交流侧相电流$(i_j)$大小,$m$是$m_j$的幅值,$\omega$是交流侧相角频率,$\theta$是交流侧功率因数角。如式(6)所示,SM电容电压的波动幅值反比例于交流频率并且正比例与交流侧相电流幅值。结果,在恒转矩应用中,SM电容器电压的波动幅值在低频时变得很大。这种情况在正交转矩应用时不那么明显因为电流幅值正比于频率。因此,有必要减小SM电容器电压的低频波动分量当MMC应用在变频恒转矩情况下时。

  \subsubsection{动态制动斩波器}
  
  在很多应用中,直流制动斩波器需要吸收和分解能量。一个例子是离岸风电场中的基于MMC的HVDC变换系统。因为MMC-HVDC末端无法接收风电场的功率,例如按上电网的非永久性故障,电网无法忍受离岸电站的任何故障。因此,如图3.8所示,岸上电站需要吸收和分解风电场的功率而在出现故障时不干扰风电场的运行。这就需要在VSC-HVDC换流站离岸直流侧安装由如图3.8所示的制动电阻来实现。
  
  另外,图3.9所示为串联连接SM的模块化设计,使模块化斩波器能够满足最小费用和空间的情况下被安装到岸上。
\begin{figure}[H]
\begin{minipage}[t]{0.5\linewidth}
\centering
\includegraphics[width=3.2in]{images/Paper_Fig_9.png}
\setcaptionwidth{2.5in}
\caption{传统制动斩波器电路}
\end{minipage}%
\begin{minipage}[t]{0.5\linewidth}
\centering
\includegraphics[width=3.2in]{images/Paper_Fig_10.png}
\setcaptionwidth{2.5in}
\caption{基于MMC概念的模块化制动斩波器}
\end{minipage}
\end{figure} 

  \subsection{结论}
  
  MMC的主要特征,例如其模块化和可测量性使其理论上满足任何电压等级的需求,有非常好的谐波性能和很高的效率。在过去几年中,对各种中/高压电压/功率系统和包括HVDC转换系统、FACTS、中压变速伺服和中/高压dc-dc换流器在内的工业应用,MMC变成了一个热点问题。
  
  对电力系统应用来说,例如HVDC系统和FACTS,MMC有一定的成熟度,并且似乎是最主流的技术,因为很多MMC-HVDC系统和STATCOMs被成功地完成和安装起来。
  
  对于中压变速伺服,有关MMC的运行和控制未来还有很多发展空间,尤其是在常转速低速的情况下。需要注意的一个主要问题是在低频情况下减小电容器电压波动的大小同时不牺牲换流器效率,因此需要在换流器大小/容量/费用和效率之间做一个权衡。
  
  源自MMC拓扑的一类模块化舵机dc-dc换流器的引入打开了中/高压dc-dc换流器领域研究和发展的一个新方向。为了更好的应用在各种应用中去,需要有高压换流器比率、高效率和器件压力较小的的高级模块化策略。
  
  随着大量出现的源自MMC的换流器拓扑和应用,可以得出结论,新的模块化和控制策略的发展会是未来MMC应用的主要推动力。
  
  
  