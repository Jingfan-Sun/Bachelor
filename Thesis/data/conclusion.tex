\chapter{结论}

\section{本论文研究总结}

本文基于现实中BPA和DIgSILENT两个电网软件各有优势但他们之间的数据不易转换的缺陷进行了研究,并提出了解决方案,即,在DPL的语言环境中对DIgSILENT进行程序编写,使得其能直接读取BPA的数据存储文件,并在DIgSILENT建立相应的数据元件和连接关系,使得在BPA中体现的数据元件和关系能够完美的移植进DIgSILENT。主要工作如下:

\begin{description}[1cm]
\item[-] 实现了不同层次的地域数据的转化。地区(ZONE)、归属地(OWNER)两个层次的地域数据,将元件数据在这些地域下进行操作,可以很直观的了解到各种元件数据之间的联系。
\item[-] 实现了B卡数据的转化。由于BPA和DIgSILENT的数据编排不同,故需要的命名方式也不同,DIgSILENT需要把BPA中的编码也加入到命名方式中,以区分双回线或者连接在同一母线上的不同元件等。且由于DIgSILENT中母线与其他元件通过开关柜(Cubic)相连,在转化过程中需要添加相应的开关柜,使得各种元件与母线的连接能够被DIgSILENT接受。同时,BPA的平衡节点是在B卡中体现,而DIgSILENT是在与此母线相连的发电机中定义,这需要编程来解决,另外BPA的B卡还包含了负载,发电机,以及补偿电容的信息,程序也实现了对这些信息的提取并且赋值于DIgSILENT相应的数据库中。
\item[-] 实现了L卡的转化。在BPA里,输电线路表示的类型很广,两个节点之间用电容器相连也是用输电线路模型来表示,而DIgSILENT里有专门的表示元件,因此,需要在编程过程中加以区别。
\item[-] 实现了T卡的转化。DIgSILENT和BPA对变压器的描述方式是不同的,BPA直接给出了阻抗数据,而DIgSILENT则是通过铁耗,铜耗等数据以及另外的计算公式去完成变压器的建模的,因此转换过程中需要注意这一点,而且变压器的分接头在两者之间也是有不同的表达方式,需要加以区分。
\end{description}

数据转化结果验证阶段,通过对IEEE39数据的潮流比较表明,该程序的数据转化是正确的。但是由于软件之间的区别,和数据转换过程中对数据位数的取舍,仍然有一些误差,这是不可避免的。

本程序经实践检验是切实可行的。综合分析这些算例的潮流比对结果,可以认为该转化程序已经可以完成基本潮流数据转化功能,是可用的。

\section{前景展望}

本文是基于IEEE39算例数据经行处理的,因此对于大电网的数据处理可能还需要完善,尤其是一些非常规性的电网,以及非常规性的数据模式的转换,程序可能还需要进一步的优化。

然而,算例以验证此程序的思路是正确的,转换后的计算结果是可行的,并且已经对各种模块数据的转换方式都提供了相应的程序模块儿,所以我相信本文提出的方法和编写的程序在已经能完成相当数据任务的基础上,通过实践的洗礼,加以改进,能够可靠的完成数据转换任务。
