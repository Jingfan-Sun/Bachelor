\chapter{开题报告}

\section{项目研究目的}
现在电力系统常用的数据格式是基于BPA文本文件,其缺陷是不易于实现自定义等高级操作,在研究新能源或新设备并网问题时,需要在现有的大电网模型基础上建立新元件模型,但 BPA 不能修改和自定义元件模型。相比之下,DIgSILENT 具有可靠灵活的系统建模能力,该软件不仅包含了其他电力系统仿真软件中的潮流、短路、机电暂态、电磁暂态等分析功能,还具有很多独特而实用的优点。它是以图形化操作和数据管理技术为支撑,便于调试和结果分析;具有风力发电、光伏发电等元件模型,也可由用户自定义各种与那件模型。

针对电力系统不同的问题,采用较优的电力系统商业软件,可以使得科研分析等工作具有事半功倍的效果。因此需要数据转换程序,以便能够选择需要的仿真软件进行仿真。

\section{研究内容和研究步骤}
研究内容主要分为以下几个方面:

\begin{itemize}
\item 复习潮流计算中所使用的主要数据类型和数据格式。
\item 学习BPA的主要功能和数据格式,BPA 数据是基于卡片格式的,主要的潮流数据卡片可以分为五类,分别为区域控制数据卡、节点数据卡、支路数据卡、变压器卡及数据修改卡。进而学习DIgSILENT的数据格式和相关操作,DIgSILENT 采用的是分层的面向对象的数据库, 潮流数据填入各个代表元件的图形中储存为对象。
\item 阅读、学习董炜学长编写的BPA到DIgSILENT的Python转化程序,在这个过程中形成自己对BPA到DIgSILENT程序转换的思路,着重理解相关参数的对应和转化,发现方法。
\item 完成程序的编写,对比分析几个软件的潮流计算结果。
\item 带入相关的算例进行测试,对比BPA和DIgSILENT的动态计算结果。
\end{itemize}

\section{程序的大体思路}

DIgSILENT自带的DPL 程序在处理以下操作中有较明显的优势: (1)实现 DIgSILENT 内部数据的导入和导出;(2) 访问或更改的 DIgSILENT 内部对象。

DPL的缺点是:不能像Python语言那样实现比较灵活和通用的功能,DPL 的开发环境不能像Python语言那样提供调试功能。因此,DPL适用于程序代码较短的功能,并不适于开发大型或复杂的功能。

因此,将DPL实现困难的部分使用Python实现,DPL易于实现的功能还是使用DPL完成,用Python整体控制转换程序的运行。主要有以下重点:(1)使用Python调用DPL程序;(2)DPL程序执行结束后,及时通知Python主程序,以便主程序执行下一步功能。

BPA 模型导入 DIgSILENT 需要以下步骤:
  \begin{enumerate}[1)]
  \item 检查 DIgSILENT 中创建的元件模型名称与 BPA 中的是否一致,给用户错误提示,直到所有错误都修改后才能继续下面步骤。可先从 DIgSILENT 导出包含电网模型的 DGS 文件,解析出元件模型名称和元件间连接关系;然后解析出 BPA 模型文件中的相关数据;最后进行对比分析。
  \item 检查 DIgSILENT 中创建的电网拓扑连接与 BPA 中的是否一致。也是通过解析对比 DGS 文件和 BPA 模型文件实现。
  \item 按照自定义的命名规则,给 DIgSILENT 中所有元件的 Foreign Key 属性定义并填入唯一的值。
  \item 将 BPA 数据文件转换为 DGS 文件。
  \item 在 DIgSILENT 中导入 DGS 文件,实现批量给元件模型赋值。
  \end{enumerate}

\section{论文的主要内容和进度安排}

本文的主要内容是通过文献资料的阅读整理,学习数据转换的基础理论知识,了解有关的常规算法以及新提出的改进算法,研究这些算法的理论并且对目前常用的算法进行收集整理,分析各种算法的优缺点以及适用条件。认识并熟悉BPA和DIgSILENT的基本使用技巧,了解他们的数据格式,编程语言,并能有一定应用。

进度安排

3月23日-4月5日:进行文献资料的阅读学习,学习软件编程语言和仿真软件,研究董炜学长的Python程序,熟悉软件使用;

4月6日-4月19日:进行程序的开发和调试;

4月20日-5月3日:结合算例对比程序运行结果,改进程序;

5月4日-结题:完成毕业论文的撰写。


